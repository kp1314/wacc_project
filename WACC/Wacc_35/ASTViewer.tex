\subsection*{Abstract Syntax Tree Viewer}

The Abstract Syntax Tree (AST) Viewer was originaly built as a way to check the 
integrity of our AST construction, the first plan being to build a graphical
viewer similar to that of ANTLR's Grun given the \texttt{-gui} flag. However after
consideration, we realised it would be far simpler and more practical to build a
static viewer that produces vector images.

The final AST Viewer works by walking over the abstract syntax tree using our
listener pattern and produces a Graphviz source in the current directory that it
then compiles with dot.

While iterating over the tree it assigns all nodes an index which used as the 
variable name in the dot source. When it is declared the \texttt{toString} of
the AST node is used to give a name and its type determines the shape of the
resulting node.

Functions are prototyped first to ensure they are defined before they are
called. We have two display modes, structured and unstructured, where structured
lays functions out horizontally with no overlapping or arcs to show calls and
unstructured places arcs on call nodes causing the Graphviz algorithm to place
functions within others. The latter allows you to see the logical structure of
the program yet the prior better relates to the generating wacc source code.

The viewer can be accessed with the \texttt{--graph} flag followed by a filename
to output to. The format can be changed with the \texttt{--format} flag, we
currently have Portable Network Graphics (PNG), Scalable Vector Graphics (SVG) 
and Postscript (PS). We default to a structured layout but unstructured graphs 
can be created by setting \texttt{--unstructured-graph}.
It is also possible with the \texttt{--graph-no-compile}
flag to leave the dot source uncompiled, full documentation and usage can be
the program with \texttt{--help}.
